%----------------------------------------
\documentclass[11pt,a4paper]{article}
%----------------------------------------
%
\usepackage{amsmath,amssymb,amsthm}
\usepackage{graphicx}
\usepackage{cancel}

%---------------------------------------------------------
\newcommand{\N}{\mathbb N}
\newcommand{\R}{\mathbb R}
\newcommand{\eps}{\varepsilon}
\newcommand{\abs}[1]{\left|#1\right|} 
\newcommand{\norm}[1]{\|#1\|}
\newcommand{\tnorm}[1]{\||#1|\|}
\newcommand{\scp}[2]{\langle#1,#2\rangle}
\newcommand{\sgn}[1]{\operatorname{sgn}(#1)}
\newcommand{\Set}[1]{\left\{#1\right\}} 
\newcommand{\SetDef}[2]{\left\{#1\;\middle|\;#2\right\}} 
\newcommand{\transpose}[1]{{#1}^{\mathsf{T}}} 
\newcommand{\dn}[1]{\frac{\partial{#1}}{\partial n}} 
\renewcommand{\div}{\operatorname{div}}
\newcommand\Rest[2]{{% we make the whole thing an ordinary symbol
  \left.\kern-\nulldelimiterspace % automatically resize the bar with \right
  #1 % the function
  \vphantom{\big|} % pretend it's a little taller at normal size
  \right|_{#2} % this is the delimiter
  }}% 
%---------------------------------------------------------
\newtheorem{theorem}{Theorem}[section]
\newtheorem{corollary}{Corollary}[theorem]
\newtheorem{lemma}[theorem]{Lemma}
\newtheorem{algorithm}[theorem]{Algorithm}
\newtheorem{remark}[theorem]{Remark}

%---------------------------------------------------------
\title{M2 MMS : Réseaux de neurones pour la modélisation}
\author{}
\date{\today}
%---------------------------------------------------------

%====================================================
\begin{document}
%====================================================

\maketitle
\tableofcontents
%
%==========================================
\section*{Introduction}\label{sec:}
%==========================================
%
%~~~~~~~~~~~~~~~~~~~~~~~~~~~~~~~~~~~~~~~~~~~~~~~~~~~~~~~~~~~~~~~~~~~~~~~~~~~~~~~~~~~~~~~
\subsection*{Objectifs}\label{subsec:}
%~~~~~~~~~~~~~~~~~~~~~~~~~~~~~~~~~~~~~~~~~~~~~~~~~~~~~~~~~~~~~~~~~~~~~~~~~~~~~~~~~~~~~~~
%
\begin{itemize}
\item Connaître les bases des machines et réseaux de neurones
\item Savoir les utiliser dans le contexte de modèles basées sur les équations différentielles ordinaires (EDO)
\item Savoir utiliser \texttt{pytorch}
\end{itemize}
%
Chaque chapitre correspond à 1-3 cours/TP.
%
%==========================================
\section{Machines et réseaux de neurones}\label{sec:}
%==========================================
%
%
%
%==========================================
\section{Fonction de perte}\label{sec:}
%==========================================
%
%
%
%==========================================
\section{Apprentissage}\label{sec:}
%==========================================
%
%
%
%==========================================
\section{EDO et méthode de collocation}\label{sec:}
%==========================================
%
%
%==========================================
\section{Réseaux pour les EDO}\label{sec:}
%==========================================
%
%
%==========================================
\section{Calcul de dérivées}\label{sec:}
%==========================================
%
%

%====================================================
\bibliographystyle{siam}
\bibliography{rndif.bib}
\end{document}
%===================================================
